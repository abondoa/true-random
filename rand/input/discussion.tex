\section{Discussion}
\label{sec:discussion}
As shown in Table~\ref{tab:tests} the throughput increases as more random numbers (bytes) are requested.
This is dues to the startup cost for initializing the generator and auxiliary object -- such as the timer itself -- before generation can start.
It is peculiar that the throughput of random numbers actually drops when the number of generated bytes is over $10^5$.
The reason is unknown, but fluctuations in workload of the test computer is a possible explanation, as the test computer is a simple desktop computer running other applications and service along with PMW.

As shown the PMW is very slow compared to a standard library implementation of a PNRG.
This is the trade-off that must be considered when deciding between an actual RNG and a PRNG.
On the other hand it is very fast compared to other true RNGs.
\texttt{/dev/random} can take up to several minutes just to generate a few hundred bytes if the pool of entropy is empty.
This means that PMW has a throughput several orders of magnitude higher that \texttt{/dev/random}.

The distribution of random bytes in Figure~\ref{fig:distribution} shows that no byte seem to appear much more often than any other.
This means that PMW exhibit equi-distribution, which is generally a desirable property of an RNG since other distributions (such as normal) can be constructed based on it.
Furthermore, it does not seem that there is a noticeable structure in which bytes are a little more represented than others.
This is desirable since it means that a potential attacker will have difficulty in guessing what is to be generated even if he has access to previously generated numbers.